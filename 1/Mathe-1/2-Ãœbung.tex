\documentclass{article}
\usepackage{enumitem}
\usepackage{amsmath,amsthm,amssymb}

\begin{document}

\title{Aufgabenblatt 2}
\author{Ben Engelhard \and Sarah Walden}
\date{}

\maketitle

\begin{enumerate}[label=\arabic*)]
    \item
          \begin{enumerate}[label=\alph*)]
              \item
                    \begin{enumerate}[label=(\roman*)]
                        \item
                              \begin{equation}
                                  \neg\forall x p(x) \equiv \exists x \neg p(x)
                              \end{equation}
                              \begin{flushleft}
                                  "p(x) gilt nicht für alle x" ist logisch äquivalent zu "es gibt ein x für das p(x) falsch ist"
                              \end{flushleft}
                        \item
                              \begin{equation}
                                  \neg\exists x p(x) \equiv \forall x \neg p(x)
                              \end{equation}
                              \begin{flushleft}
                                  "Es gibt kein x bei dem p(x) wahr ist" ist logisch äquivalent zu "für alle x ist p(x) falsch"
                              \end{flushleft}
                    \end{enumerate}
              \item
                    \begin{align}
                        \nonumber\neg\forall x \in X \exists y\in Y : p(x,y) \wedge q(x,y)            \\
                        \nonumber\equiv \exists x \in X \neg (\exists y \in Y : p(x,y) \wedge q(x,y)) \\
                        \nonumber\equiv \exists x \in X \forall y \in Y : \neg (p(x,y) \wedge q(x,y)) \\
                        \equiv \exists x \in X \forall y \in Y : \neg p(x,y) \vee \neg q(x,y)
                    \end{align}
              \item \begin{enumerate}[label=(\roman*)]
                        \item \begin{equation}
                                  \exists x p(x) \vee \exists x q(x) \equiv \exists x p(x) \vee q(x)
                              \end{equation}
                              \begin{proof}
                                  \begin{align*}
                                      \exists x p(x) \vee \exists x q(x) \equiv \exists x p(x) \vee q(x)                              \\
                                      \Leftrightarrow \neg( \exists x p(x) \vee \exists x q(x)) \equiv \neg(\exists x p(x) \vee q(x)) \\
                                      \Leftrightarrow \forall x \neg p(x) \wedge \forall x \neg q(x) \equiv \forall x \neg p(x) \wedge \neg q(x)
                                  \end{align*}
                                  "$\Rightarrow$"
                                  \begin{align*}
                                      \forall x \neg p(x) \wedge \forall x \neg q(x) \\
                                      \Rightarrow \neg p(x), \neg q(x)               \\
                                      \Rightarrow \forall x \neg p(x) \wedge \neg q(x)
                                  \end{align*}
                                  "$\Leftarrow$"
                                  \begin{align*}
                                      \forall x \neg p(x) \wedge \neg q(x) \\
                                      \Rightarrow \neg p(x), \neg q(x)     \\
                                      \forall x \neg p(x) \wedge \forall x \neg q(x)
                                  \end{align*}
                              \end{proof}
                        \item \begin{equation}
                                  \exists x p(x) \wedge \exists q(x) \equiv \exists x p(x) \wedge q(x)
                              \end{equation}
                              \begin{proof}[Wiederlegen durch Gegenbeispiel]

                                  Sei M=\{a,b\} und p(a) = w, q(a) = f, p(b) = f, q(b) = w \\
                                  Dann ist $\exists x p(x)$ wahr, $\exists x q(x)$ wahr und $\exists x p(x) \wedge \exists x p(x)$ wahr.\\
                                  $\exists x p(x) \wedge q(x)$ ist aber falsch, da entweder immer p(x) oder q(x) für jedes Element falsch ist. \\
                                  d.h. die beiden Seiten der Aussage sind nicht äquivalent
                              \end{proof}
                    \end{enumerate}
          \end{enumerate}
    \item \begin{enumerate}[label=\alph*)]
              \item $ A \cup B = \{a,b,c,d\}$
              \item $ A \cap B = \{b,c\}$
              \item $ A \backslash B = \{a\}$
              \item $ (A \cup B)\backslash (A \cap B) = \{a,d\}$
              \item $A \times B$ = \{(a,b),(a,c),(a,d),(b,b),(b,c),(b,d),(c,b),(c,c),(c,d) \}
              \item $P(A) = \{\{a,b,c\},\{a,b\},\{a,c\},\{b,c\},\{a\},\{b\},\{c\},\emptyset\}$
              \item $A \cap C = \emptyset$
              \item $P(\emptyset) = \{\emptyset\}$
          \end{enumerate}
    \item \begin{enumerate}[label=\alph*)]
              \item \begin{equation}
                        P(A \cap B) =P(A) \cap P(B)
                    \end{equation}
                    \begin{proof}[Beweis durch Fallunterscheidung]
                        Fall 1: $A \cap B = \emptyset$
                        \begin{gather*}
                            P(A) \cap P(B) = \{\emptyset\} \\
                            P(A \cap B) = \{\emptyset\}
                        \end{gather*}\hfill \checkmark

                        Fall 2: $A = B$
                        \begin{gather*}
                            A \cap B = A = B \\
                            P(A \cap B) = P(A) = P(B)
                            = P(A) \cap P(B)
                        \end{gather*} \hfill \checkmark

                        Fall 3: $(A \cap B) \subset A \wedge (A \cap B) \subset B$
                        \begin{gather*}
                            P(A \cap B) = \{C |C \subset A \cap B\} \\
                            P(A) = \{A^{*} | A^{*} \subset A \} \\
                            P(B) = \{B^{*} | B \subset B \} \\
                        \end{gather*}


                    \end{proof}
                    \pagebreak
              \item \begin{equation}
                        P(A \cup B) =P(A) \cup P(B)
                    \end{equation}
                    \begin{proof}[Wiederlegen durch Gegenbeispiel]
                        Sei A=\{1\} und B=\{2\}\\
                        Dann ist:
                        \begin{gather*}
                            P(A \cup B) = \{\emptyset, \{1\},\{2\},\{1,2\}\} \\
                            P(A) = \{\emptyset, \{1\}\}                      \\
                            P(B) = \{\emptyset, \{2\}\}                      \\
                            P(A) \cup P(B) = \{\emptyset, \{1\}, \{2\}\}     \\
                        \end{gather*}
                        $P(A \cup B) \not= P(A) \cup P(B) $
                    \end{proof}
          \end{enumerate}

    \item \begin{equation}
              \exists p^{\star}T(p^{\star}) \Rightarrow \forall p T(p)
          \end{equation}
          \begin{proof}[Wiederlegen durch Gegenbeispiel]
              \begin{displaymath}
                  \begin{array}{|c c | c |}
                      A & B & A \Rightarrow B \\
                      \hline
                      w & w & w               \\
                      w & f & f               \\
                      f & w & w               \\
                      f & f & w               \\
                  \end{array}
              \end{displaymath}
              \paragraph{Fall 1: Alle $p \in P$ trinken:}
              \begin{gather*}
                  \exists p^{\star} T(^{\star}) = wahr \\
                  \forall p T(p) = wahr \\
                  \exists p^{\star} T(^{\star}) \Rightarrow  \forall p T(p) = wahr \\
                  \checkmark
              \end{gather*}
              \paragraph{Fall 2: Eine $p \in P$ trinkt nicht:} Wenn $\exists p \in P$ nicht trinkt, ist $\forall p \in P$ falsch, obwohl $\exists p^{\star} T(^{\star})$ immer noch wahr ist.
              \begin{equation}
                  \exists p^{\star} T(^{\star}) \nRightarrow  \forall p T(p)
              \end{equation}
          \end{proof}
\end{enumerate}

\end{document}